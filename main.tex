%%%%%%%%%%%%%%%%%%%%%%%%%%%%%%%%%%%%%%%%%
% Masters/Doctoral Thesis 
% LaTeX Template
% Version 2.5 (27/8/17)
%
% This template was downloaded from:
% http://www.LaTeXTemplates.com
%
% Version 2.x major modifications by:
% Vel (vel@latextemplates.com)
%
% This template is based on a template by:
% Steve Gunn (http://users.ecs.soton.ac.uk/srg/softwaretools/document/templates/)
% Sunil Patel (http://www.sunilpatel.co.uk/thesis-template/)
%
% Template license:
% CC BY-NC-SA 3.0 (http://creativecommons.org/licenses/by-nc-sa/3.0/)
%
%%%%%%%%%%%%%%%%%%%%%%%%%%%%%%%%%%%%%%%%%

%----------------------------------------------------------------------------------------
%	PACKAGES AND OTHER DOCUMENT CONFIGURATIONS
%----------------------------------------------------------------------------------------

\documentclass[
11pt, % The default document font size, options: 10pt, 11pt, 12pt
%oneside, % Two side (alternating margins) for binding by default, uncomment to switch to one side
french, % ngerman for German
singlespacing, % Single line spacing, alternatives: onehalfspacing or doublespacing
%draft, % Uncomment to enable draft mode (no pictures, no links, overfull hboxes indicated)
%nolistspacing, % If the document is onehalfspacing or doublespacing, uncomment this to set spacing in lists to single
%liststotoc, % Uncomment to add the list of figures/tables/etc to the table of contents
%toctotoc, % Uncomment to add the main table of contents to the table of contents
%parskip, % Uncomment to add space between paragraphs
%nohyperref, % Uncomment to not load the hyperref package
headsepline, % Uncomment to get a line under the header
%chapterinoneline, % Uncomment to place the chapter title next to the number on one line
%consistentlayout, % Uncomment to change the layout of the declaration, abstract and acknowledgements pages to match the default layout
]{MastersDoctoralThesis} % The class file specifying the document structure
\usepackage{float}
\usepackage[utf8]{inputenc} % Required for inputting international characters
\usepackage[T1]{fontenc} % Output font encoding for international characters
%\DeclareUnicodeCharacter{0301}{*************************************}
\usepackage{babel}
\usepackage[autolanguage]{numprint}

\usepackage{mathpazo} % Use the Palatino font by default

\usepackage[backend=bibtex,style=authoryear,natbib=true]{biblatex} % Use the bibtex backend with the authoryear citation style (which resembles APA)

\addbibresource{example.bib} % The filename of the bibliography

\usepackage[autostyle=true]{csquotes} % Required to generate language-dependent quotes in the bibliography
%\DeclareUnicodeCharacter{0301}{\'{e}}
%----------------------------------------------------------------------------------------
%	MARGIN SETTINGS
%----------------------------------------------------------------------------------------

\geometry{
	paper=a4paper, % Change to letterpaper for US letter
	inner=2.5cm, % Inner margin
	outer=2.8cm, % Outer margin
	bindingoffset=.5cm, % Binding offset
	top=1.5cm, % Top margin
	bottom=1.5cm, % Bottom margin
	%showframe, % Uncomment to show how the type block is set on the page
}

%----------------------------------------------------------------------------------------
%	THESIS INFORMATION
%----------------------------------------------------------------------------------------
 

\thesistitle{Rapport de stage de fin de cycle} % Your thesis title, this is used in the title and abstract, print it elsewhere with \ttitle
\supervisor{Dr. Yehia \textsc{TAHER}} % Your supervisor's name, this is used in the title page, print it elsewhere with \supname
\examiner{} % Your examiner's name, this is not currently used anywhere in the template, print it elsewhere with \examname
\degree{Doctor of Philosophy} % Your degree name, this is used in the title page and abstract, print it elsewhere with \degreename
\author{Mr. Mickael \textsc{BENHASSEN}} % Your name, this is used in the title page and abstract, print it elsewhere with \authorname
\addresses{} % Your address, this is not currently used anywhere in the template, print it elsewhere with \addressname

\subject{Biological Sciences} % Your subject area, this is not currently used anywhere in the template, print it elsewhere with \subjectname
\keywords{} % Keywords for your thesis, this is not currently used anywhere in the template, print it elsewhere with \keywordnames
\university{\href{https://www.uvsq.fr}{Université Paris Saclay}} % Your university's name and URL, this is used in the title page and abstract, print it elsewhere with \univname
\department{\href{https://www.uvsq.fr}{dieu-merci.kimpolo-nkokolo@ens.uvsq.fr}} % Your department's name and URL, this is used in the title page and abstract, print it elsewhere with \deptname
\group{\href{https://www.uvsq.fr}{Dieu Merci KIMPOLO NKOKOLO}} % Your research group's name and URL, this is used in the title page, print it elsewhere with \groupname
\faculty{\href{https://www.uvsq.fr}{Faculty Name}} % Your faculty's name and URL, this is used in the title page and abstract, print it elsewhere with \facname

\AtBeginDocument{
\hypersetup{pdftitle=\ttitle} % Set the PDF's title to your title
\hypersetup{pdfauthor=\authorname} % Set the PDF's author to your name
\hypersetup{pdfkeywords=\keywordnames} % Set the PDF's keywords to your keywords
}

\begin{document}
\frontmatter % Use roman page numbering style (i, ii, iii, iv...) for the pre-content pages

\pagestyle{plain} % Default to the plain heading style until the thesis style is called for the body content

%----------------------------------------------------------------------------------------
%	TITLE PAGE
%----------------------------------------------------------------------------------------

\begin{titlepage}
\begin{center}


%---------------------------------
%. Logos
%--------------------------------
  \begin{center}
  \setlength{\tabcolsep}{0pt}
  \begin{tabular}{>{\raggedleft}m{2.5cm}>{\centering}m{\dimexpr\textwidth - 5cm\relax}>{\raggedright}m{2.5cm}}
  \includegraphics[width=\linewidth]{Figures/giga}%
  &%
  %\textbf{\large INSTITUTO POLITECNICO NACIONAL} \\[5pt]%
  %\textbf{\large ESCUELA SUPERIOR DE COMPUTO}%
  &%
  \includegraphics[width=\linewidth]{Figures/logouvsq} %
  \end{tabular}
  \end{center}

\vspace*{.06\textheight}
{\scshape\LARGE \univname\par}\vspace{1.5cm} % University name
\textsc{\Large Master Datascale}\\[0.5cm] % Thesis type

\HRule \\[0.4cm] % Horizontal line
{\huge \bfseries \ttitle\par}\vspace{0.4cm} % Thesis title
\HRule \\[1.5cm] % Horizontal line
 
\begin{minipage}[t]{0.4\textwidth}
\begin{flushleft} \large
\emph{Maître de stage:}\\
\href{http://www.johnsmith.com}{\authorname} % Author name - remove the \href bracket to remove the link
\end{flushleft}
\end{minipage}
\begin{minipage}[t]{0.4\textwidth}
\begin{flushright} \large
\emph{Superviseur:} \\
\href{http://www.jamessmith.com}{\supname} % Supervisor name - remove the \href bracket to remove the link  
\end{flushright}
\end{minipage}\\[3cm]
 
\vfill

%\large \textit{A thesis submitted in fulfillment of the requirements\\ for the degree of \degreename}\\[0.3cm] % University requirement text
%\textit{in the}\\[0.4cm]
\groupname\\\deptname\\[2cm] % Research group name and department name
 
\vfill

{\large \today}\\[4cm] % Date
%\includegraphics{Logo} % University/department logo - uncomment to place it
 
\vfill
\end{center}
\end{titlepage}

%----------------------------------------------------------------------------------------
%	DECLARATION PAGE
%----------------------------------------------------------------------------------------
\chapter *{Remerciements}
Premièrement je remercie tout d’abord le Dieu tout Puissant pour le souffle de vie qu’il nous procure au quotidien et sa grâce .\\\\Je remercie également toute l'équipe  technique de Gigamesh, en particulier mon encadreur professionnel Mickaël BENHASSEN, pour sa disponibilité et son aide permanente.\\ \\
Je tiens à remercier aussi, toute l’équipe pédagogique de l’Université Paris Saclay qui a assuré avec dévouement ma formation de Master DataScale et particulièrement Monsieur TAHER Yehia mon encadrant durant ce stage.\\ \\Enfin je voudrais aussi remercier très sincèrement ma famille pour avoir été toujours à mes côtés durant mes études.
%Enfin je témoigne toute ma gratitude à toutes les personnes qui n’ont cessé de me soutenir toutau long de ma formation.



%J’exprime également mes remerciements à tous mes promotionnaires, mes amis et mes proches pour leur attention, soutien et accompagnement.
%Enfin je voudrais aussi remercier très sincèrement ma famille pour avoir été toujours à mes côtés durant mes études. Je le fais particulièrement à mon Papa AZOTSIE Ferdinand, ma Maman AZOTSIE Christine, tous mes oncles, mes frères et mes sœurs de la famille Magui, ma réussite et mon succès n’ont tenu qu’à votre soutien, vos conseils et vos prières.
\chapter*{Introduction}
%\begin{justify}
L’assurance de prêt généralement désignée par assurance emprunteur, est une garantie demandée par les prêteurs (les banques) lors d’une demande de prêt. 
Bien que ce ne soit pas une obligation légale en France, elle est exigée dans la quasi-totalité des cas. Cette assurance permet de couvrir les risques de défaut 
de paiement quelles que soient leurs causes, ce qui explique qu’elle soit ainsi exigée.\\ \\
En France, en  2017,  le  chiffre  d’affaire  de  l’assurance  emprunteur  (crédit  à  la consommation, prêt immobilier et prêt professionnel) était de 9 milliards d’euros dont 6  milliards  d’euros  provenant  uniquement  de  l’assurance  emprunteur  des  prêts immobiliers. Ces chiffres, sont toujours  en augmentation. Et près de 50\% du marché de l'assurance emprunteur est contrôlé par deux acteurs: CNP Assurances et   Crédit   Agricole.\\ 
Un petit nombre d'entreprises a le monopole du marché de l'assurance emprunteur en France, ce qui fait que le secteur manque d'innovation en terme d'amelioration des prix, de transparence et de simplicité des procedures de souscription. Aussi les assurances emprunteur traditionnelles ont construit leur analyse des risques sur une photographie figée de l’assuré, en décalage avec les nombreux changements que la vie réserve. Ils ont donc perdu de vue leur mission de départ : protéger en cas de problème. %Le secteur manque de l'innovation.%Tout est parti du constat que les assurances emprunteur traditionnelles ont construit leur analyse des risques sur une photographie figée de l’assuré, en décalage avec les nombreux changements que la vie réserve. Ils ont donc perdu de vue leur mission de départ : protéger en cas de problème.
%Le secteur de l'assurance emprunteur peine à s'aligner par rapport aux atouts technologiques de notre époque.\\ \\
%Le domaine de l'assurance manque de l'innovation en terme technologique, d'amelioration des prix et de simplicité de procedure de souscription Gigamesh compte donner du pouvoir au client.\\ \\
GigaMesh avec son produit d'assurance emprunteur Assurly compte  révolutionner le domaine de l'assurance emprunteur en associant à son expertise assurantielle  la transparence et les atouts technologiques de notre époque. \\ 
Pour GigaMesh l'assurance doit s'adapter aux moments de la vie des clients, tout au long de leur vie. En toute transparence, en cherchant 
à constamment simplifier la vie du client. Les technologies de pointe: Edge-computing, Machine learning, Cloud, Serveless, apis Data science
doivent servir le client. \\ \\
C'est pour participer à cette revolution et pour me  permettre de compléter mon Master 2 Datascale qui exige un stage de 6 mois en entreprise que GigaMesh m'avait offert un stage de six mois dans son service IT comme développeur Backend.\\ L’essentiel de ce document sera constitué du développement dans l’ordre cité des points suivants: la
présentation de l’entreprise d’accueil, un état de l'art par rapport aux missions qui m'avaient été confiées, la présentation des technologies utilisées, et la présentation de la méthodologie de travail et des solutions proposées  par rapport à mes missions .
 
%\end{justify}
%\begin{declaration}
%\addchaptertocentry{\authorshipname} % Add the declaration to the table of contents
%\noindent I, \authorname, declare that this thesis titled, \enquote{\ttitle} and the work presented in it are my own. I confirm that:

%\begin{itemize} 
%\item This work was done wholly or mainly while in candidature for a research degree at this University.
%\item Where any part of this thesis has previously been submitted for a degree or any other qualification at this University or any other institution, this has been clearly stated.
%\item Where I have consulted the published work of others, this is always clearly attributed.
%\item Where I have quoted from the work of others, the source is always given. With the exception of such quotations, this thesis is entirely my own work.
%\item I have acknowledged all main sources of help.
%\item Where the thesis is based on work done by myself jointly with others, I have made clear exactly what was done by others and what I have contributed myself.\\
%\end{itemize}
 
%\noindent Signed:\\
%\rule[0.5em]{25em}{0.5pt} % This prints a line for the signature
 
%\noindent Date:\\
%\rule[0.5em]{25em}{0.5pt} % This prints a line to write the date
%\end{declaration}

%\cleardoublepage

%----------------------------------------------------------------------------------------
%	QUOTATION PAGE
%----------------------------------------------------------------------------------------

\vspace*{0.2\textheight}

%\noindent\enquote{\itshape Thanks to my solid academic training, today I can write hundreds of words on virtually any topic without possessing a shred of information, which is how I got a good job in journalism.}\bigbreak

%\hfill Dave Barry

%----------------------------------------------------------------------------------------
%	ABSTRACT PAGE
%----------------------------------------------------------------------------------------
\chapter*{Résumé}
%\begin{abstract}
GigaMesh est une entreprise oeuvrant dans le domaine de l'assurance emprunteur. GigMesh ambitionne de révolutionner le domaine 
de l'assurance en proposant un produit d'assurance simple accessible, à juste prix et aligné aux technologies de notre époque. \\ \\
J'ai passé mon stage chez Gigamesh comme développeur Backend.  Mon travail portait principalement sur des questions liées: au déploiement et l'intégration de microservices, à l'intégration des données, à la sécurisation de l'API et au  déploiement des applications web dans l'environnement AWS. \\ \\
 J'avais ainsi  amélioré et acquis les compétences techniques liées: au  développement et à la sécurisation
des API dans l'environnement AWS, à intégration de données dans Mysql avec Aurora, à la construction et au déploiement des conteneur Docker
et aussi à l’utilisation des machines virtuelles dans le Cloud.\\ \\
Mon stage chez  Gigamesh m'avait également permis de découvrir le domaine d'assurance, d’acquérir des compétences en gestion de projet avec la méthodologie de travail Scrum et dans l’utilisation de certains outils tel que GitHub et Reflex.

%Mon stage chez GigaMesh était non seulement un moment d’apprentissage mais également d’application.

%\addchaptertocentry{\abstractname} % Add the abstract to the table of contents
%The Thesis Abstract is written here (and usually kept to just this page). The page is kept centered vertically so can expand into the blank space above the title too\ldots
%\end{abstract}

%J’ai effectué mon stage auprès de la startup Gigamesh ayant pour projet de se lancer dans l’assurance emprunteur. Mon rôle aura été de participer à l’élaboration du produit d’assurance « Assurly by Gigamesh » à travers le développement de l’application mobile essentielle au projet.
%Afin de répondre aux besoins de l’entreprise, j’ai appris à utiliser le framework Flutter permettant de réaliser des applications mobiles. Mes missions sont passées de la réalisation de la partie front jusqu’aux échanges de données avec le back, en passant par des phases de recherches et d’échanges en vue d’améliorer le produit.
%Travailler avec Gigamesh m’aura également permis d’acquérir certaines compétences, notamment en gestion de projet avec la méthodologie de travail Scrum et dans l’utilisation de certains outils tel que GitHub. Voir l’évolution du projet Assurly restera pour moi une expérience exceptionnelle et riche en apprentissage.
%----------------------------------------------------------------------------------------
%	ACKNOWLEDGEMENTS
%----------------------------------------------------------------------------------------

%\begin{acknowledgements}
%Remerciements
%\addchaptertocentry{Remerciements:} % Add the acknowledgements to the table of contents
%The acknowledgments and the people to thank go here, don't forget to include your project advisor\ldots
%\end{acknowledgements}
%\chapter*{Remerciements}
%Ce stage aura été pour moi une expérience exceptionnelle et ce grâce à l’ensemble de l’équipe Assurly que je remercie.
%Je souhaite bien évidemment à Assurly de réussir dans son lancement et de réaliser cette vision de l’assurance emprunteur de demain.
%Je tiens à remercier toute l’équipe tech qui su mettre en place un environnement de travail efficace et agréable notamment grâce à Harry Belinga, mon tuteur. Je remercie tout particulièrement Mickaël Benhassen et Pierre Gancel qui m’auront énormément appris en développement mobile.
%Enfin, je remercie l’UTT qui m’a permis de réaliser ce stage qui complète parfaitement les enseignements que j’ai suivi jusqu’à maintenant.
%----------------------------------------------------------------------------------------
%	LIST OF CONTENTS/FIGURES/TABLES PAGES
%----------------------------------------------------------------------------------------

\tableofcontents % Prints the main table of contents

\listoffigures % Prints the list of figures

%\listoftables % Prints the list of tables

%----------------------------------------------------------------------------------------
%	ABBREVIATIONS
%----------------------------------------------------------------------------------------
\chapter*{Abreviations}

%\textbf{Liste des abréviations:} 
 \begin{figure}[H]
            \centering
                \includegraphics[width=0.8\textwidth]{Figures/abrevia}
	       \decoRule
		%\caption[Abreviations]{Abreviations}
	\label{fig:Abreviations}
\end{figure}

%\end{abbreviations}

%----------------------------------------------------------------------------------------
%	PHYSICAL CONSTANTS/OTHER DEFINITIONS
%----------------------------------------------------------------------------------------

%\begin{constants}{lr@{${}={}$}l} % The list of physical constants is a three column table

% The \SI{}{} command is provided by the siunitx package, see its documentation for instructions on how to use it

%Speed of Light & $c_{0}$ & \SI{2.99792458e8}{\meter\per\second} (exact)\\
%Constant Name & $Symbol$ & $Constant Value$ with units\\

%\end{constants}

%----------------------------------------------------------------------------------------
%	SYMBOLS
%----------------------------------------------------------------------------------------

%\begin{symbols}{lll} % Include a list of Symbols (a three column table)

%$a$ & distance & \si{\meter} \\
%$P$ & power & \si{\watt} (\si{\joule\per\second}) \\
%Symbol & Name & Unit \\

%\addlinespace % Gap to separate the Roman symbols from the Greek

%$\omega$ & angular frequency & \si{\radian} \\

%\end{symbols}

%----------------------------------------------------------------------------------------
%	DEDICATION
%----------------------------------------------------------------------------------------

%\dedicatory{For/Dedicated to/To my\ldots} 

%----------------------------------------------------------------------------------------
%	THESIS CONTENT - CHAPTERS
%----------------------------------------------------------------------------------------

\mainmatter % Begin numeric (1,2,3...) page numbering

\pagestyle{thesis} % Return the page headers back to the "thesis" style

% Include the chapters of the thesis as separate files from the Chapters folder
% Uncomment the lines as you write the chapters

% Chaptre 1

\chapter{Contexte du stage} % Main chapter title

\label{Chaptre1} % For referencing the chapter elsewhere, use \ref{Chapter1} 

Dans cette section je vais faire une présentation générale de Gigamesh, ses missions, et ses projets.

%----------------------------------------------------------------------------------------

% Define some commands to keep the formatting separated from the content 
\newcommand{\keyword}[1]{\textbf{#1}}
\newcommand{\tabhead}[1]{\textbf{#1}}
\newcommand{\code}[1]{\texttt{#1}}
\newcommand{\file}[1]{\texttt{\bfseries#1}}
\newcommand{\option}[1]{\texttt{\itshape#1}}
%----------------------------------------------------------------------------------------

\section{Présentation de Gigamesh }

Fondée en 2017 par Toufik Gozim et Mickaeel Benhassen, Gigamesh est une InsurTech française visant à se lancer dans l’assurance emprunteur. Pour ce faire, l’entreprise travaille depuis sa creéation à obtenir l’agreément deélivré par l’ACPR (Autorité de Contrôle Prudentiel et de Résolution), nécessaire pour proposer une assurance. Cette institution est chargée de surveiller les banques et assurances en France.				
\\

Gigamesh  devrait obtenir l’agrément en fin d’année 2021 et pourra donc officiellement lancer son produit d’assurance. Le choix de se positionner dans le domaine de l’assurance emprunteur n’est pas anodin. En effet, ce secteur est un oligopole détenu par des géants avec peu de place à la concurrence. Nous verrons par la suite sa complexité mais aussi l’intérêt qu’il suscite.

La loi Bourquin de février 2017, ouvre le marché de l'assurance emprunteur en permettant aux assurés de pouvoir changer leur contract d'assurance chaque année à la date anniversaire. De plus, le préavis pour effectuer ce changement est passé de 15 jours(Loi Hamon) à 2 mois. L'idée prend forme, puis le projet devient concret en octobre 2017 avec le création de la société Gigamesh SAS par Toufik Et Michael.
\\	
Gigamesh souhaite révolutionner le secteur de l’assurance emprunteur sur tous les fronts. Redonner du pouvoir d’achat, apporter une approche technologique et développer de nouvelles relations clients sont au cœur de leur vision. 



%----------------------------------------------------------------------------------------

\section{L’organigramme de Gigamesh :}
 \begin{figure}[!th]
            \centering
                \includegraphics[width=0.8\textwidth]{Figures/orga}
	       \decoRule
		\caption[Organigramme]{Organigramme}
	\label{fig:Organigramme}
\end{figure}
 \textbf{A compléter}

\section{Localisation de Gigamesh :}

Localisé à Paris dans un espace de travail: \textbf{platform58}, \textbf{Gigamesh } profite d’un environnement de coworking et de bureaux privés. Un espace de coworking est un espace de travail partage mettant en avant l’échange et l’ouverture aux autres. Située au \textbf{58, rue de la Victoire 75009 Paris}, la \textbf{platform58} est un lieu d’innovation de la Banque Postale qui
repose sur deux briques :
\begin{itemize}
	\item Un programme d’accompagnement de startups
	\item Un lieu physique proposant la location d’espaces de travail
\end{itemize}

\section{Domaine d’activité de Gigamesh}
Nous allons voir dans cette partie le domaine dans lequel évolue \textbf{Gigamesh} et la réponse apportée pour le faire progresser.
\subsection{Assurance emprunteur}


L’assurance de prêt généralement désignée par assurance emprunteur (article L. 313-29 du code de la
consommation) est une garantie demandée par les prêteurs (les banques) lors d’une demande de prêt.
Bien que ce ne soit pas une obligation légale, elle est exigée dans la quasi-totalité des cas. Cette assurance permet de couvrir les risques de défaut de paiement quelles que soient leurs causes, ce qui explique qu’elle soit ainsi exigée. Elle comporte des garanties couvrant les risques :
\begin{itemize}
	\item D’incapacité
	\item D’invalidité
	\item Voire de perte d’emploi
\end{itemize}

\begin{figure}[!th]
\centering
\includegraphics[width=0.8\textwidth]{Figures/emprunteur}
\decoRule
\caption[Illustration du système d'emprunt]{Illustration du système d'emprunt.}
\label{fig:Emprunt}
\end{figure}

\subsection{Assurly}
Assurly est le nom donne au produit d’assurance de Gigamesh  et represente tout le travail effectue par
l’entreprise depuis sa création. Assurly est donc le résultat du projet de Gigamesh 
Assurly est une assurance emprunteur qui bouleverse le secteur de l’assurance en créant un produit clair,
simple et au juste prix.

Tout est parti du constat que les assurances emprunteur traditionnelles ont construit leur analyse des risques sur une photographie figée de l’assuré, en décalage avec les nombreux changements que la vie réserve. Ils ont donc perdu de vue leur mission de départ : protéger en cas de problème.

Suite à 3 ans de recherche et développement, Gigamesh a réussi à associer les apports de l’innovation
technologique et leur expertise assurantielle pour créer Assurly.

Assurly est née avec l’ambition de redonner le pouvoir aux assurés en proposant un produit clair avec des
garanties all-inclusive au meilleur tarif. Assurly transforme l’assurance emprunteur grâce à une expérience client simplifiée et 100\% digitale : une souscription en moins de 10 minutes depuis son portable avec zéro paperasse, zéro rendez-vous, zéro stress !
Assurly propose de couvrir ces risques avec les garanties suivantes :

\begin{enumerate}
	\item La garantie perte totale et irréversible d’autonomie (PTIA).\\
	
Elle intervient lorsque l’assuré se trouve dans un état particulièrement grave, nécessitant le recours permanent à une tierce personne pour exercer les actes ordinaires de la vie.\\

\textbf{La couverture Assurly} : Gigamesh sera là pour vous épauler et rembourser 100\% des mensualités du reste de votre prêt. La garantie PTIA cesse au 71ème anniversaire de l’assuré.

\item La garantie incapacité temporaire totale (ITT) dénommée incapacité temporaire totale dans le contrat.\\\\

Elle intervient lorsque la personne assurée est temporairement inapte à exercer son activité professionnelle ou, si il n’en a pas, d’observer un repos complet l’obligeant à interrompre toutes ses occupations de la vie
quotidienne\\

\textbf{La couverture Assurly} : Gigamesh 
h seralà pour vous épauler et payer 100\% vos mensualités de remboursement de prêt durant votre temps d’impossibilité de travailler avec un plafond de 7500 euros par mois, quelle que soit votre perte de revenu. La garantie ITT cesse au plus tard au jour du 65ème anniversaire de l’assuré. Les affections dorsales, psychiques et psychiatriques causant l’ITT sont couvertes sans condition d’hospitalisation
ou d’intervention chirurgicale.
\begin{list}{label}{spacing}
	\item La garantie Invalidité prend deux formes :
	 \begin{enumerate}
	 	\item La garantie invalidité permanente partielle (IPP) :\\
	 	Elle intervient lorsque la personne assurée est, de façon définitive, incapable d’exercer strictement son
	 	activité après la reconnaissance de l’état d’invalidité estimé avec un taux entre 33\% et 66\%.\\
	 	
	 	\textbf{La couverture Assurly }: Gigamesh  sera là pour vous épauler et payer 50\% de l’indemnité garantie en cas d’ITT, quelle que soit votre perte de revenu, avec un plafond de 7500€ par mois. Le délai de franchise maximale est de 90 jours après l’interruption de l’activité. Les affections dorsales, psychiques et psychiatriques causant l’IPP sont couvertes sans condition d’hospitalisation
	 	ou d’intervention chirurgicale
	 	\item La garantie Invalidité Permanente Totale (I.P.T.)\\
	 	Elle intervient lorsque la personne assurée est, de
	 	façon définitive, incapable d’exercer strictement son activité professionnelle après la reconnaissance de l’état d’invalidité estimé avec un taux supérieur à 65\%\\
	 	
	 	\textbf{La couverture Assurly} : Gigamesh  sera là pour vous épauler et rembourser 100\% des mensualités du reste de votre prêt avec un plafond de 3 000 000 euros, quelle que soit votre perte de revenu. La garantie invalidité cesse au jour du 65ème anniversaire de l’assuré.
	 \end{enumerate}

\end{list}
\item La garantie décès\\
Elle intervient en cas de décès de la personne assurée. Nous serons là pour épauler votre famille et payer 100\% de vos mensualités de remboursement de votre prêt.

\end{enumerate}

\section{Objectif du stage}
Cette partie fera l’objet de la présentation du projet sur lequel j’avais travaillé et de mes propres mission dans ce celui ci:
Assurly est une plateforme digitale et innovante, qui simplifie le système d’assurance d’emprunt.
Elle est constituée des modules suivante:
\begin{itemize}
\item  Une application mobile B2C, pour la simulation de l’assurance et l’espace client,
\item  Une application B2B, pour la simulation de l’assurance et de la création d’un assure par des partenaires,
\item Un ensemble d’application de gestion, paiement en ligne et service client,
\item Une application backend pour la gestion et données et la fourniture des APIs
\end{itemize}

Les applications frontend communiquent avec le backend via une API de type Rest.
Dans le cadre de mon stage j’intervenais plus souvent dans le backend et sur les missions suivantes:

\begin{list}{•}
	\item Le déploiement et l’intégration d’un service web (Reflex) dans un environnement AWS
	\item Intégration des données en format txt dans une base de données sql dans l’environnement AWS
	\item Etude de faisabilité sur le déploiement d’une application Front-End dans un environnement AWS
	\item Sécurisation d’une API Rest développée avec la technologie Serveless d’AWS (Lambada + python).
	\item Le déploiement d’un système de web scraping dans un environnement AWS.
\end{list}
\begin{figure}[!th]
\centering
\includegraphics[width=0.8\textwidth]{Figures/architecture}
\decoRule
\caption[L'architecture]{L'architecture}
\label{fig:architecture}
\end{figure}


%----------------------------------------------------------------------------------------




% Chaptre 1

\chapter{Etat de l’art} % Main chapter title

\label{Chaptre3} % For referencing the chapter elsewhere, use \ref{Chapter1} 

Dans cette partie nous présenterons le positionnement du travail demandé par rapport à l’état de l’art scientifique ou technique.

\section{Les APIs(Application programming interface)}

une API est une façade clairement délimitée par laquelle une application offre des services à d’autres applications. Cette façade peut comporter des classes, des méthodes ou des fonctions, des types de données et
des constantes. Les API peuvent être publiques, partenaires ou privées
 \begin{figure}[!th]
            \centering
                \includegraphics[width=0.8\textwidth]{Figures/api}
	       \decoRule
		\caption[API]{API}
	\label{fig:api}
	\end{figure}
\subsection{Objectif d’une API}
Tout comme une interface utilisateur graphique facilite l’utilisation des programmes pour les profanes, une
API a pour objectif de fournir une porte d’accès à une ou plusieurs fonctionnalités d’un système en exposant
uniquement les objets ou les actions dont le développeur a besoin et en cachant les détails de la mise en
œuvre. Elles permettent ainsi aux développeurs d’utiliser plus facilement certaines technologies pour créer
des applications. Une API simplifie la programmation.



\subsection{Analogie d’une API avec la vie courante}

Prenons l’exemple d’un appel téléphonique, la plupart des gens ne savent pas comment se déroule un appel
téléphonique : le mécanisme électronique qui régit un appel téléphonique. Pourtant tout le monde peut
effectuer un appel juste en composant le numéro de son correspondant et en appuyant simplement sur
un bouton (envoyer ou ok) du clavier. Le clavier et l’écran du téléphone sont l’équivalent de l’API; ils
représentent l’interface du système électronique qui régit un appel téléphonique

\section{Le modèle ou l'architecture Rest}
Les APIs peuvent être intégrées suivant plusieurs modèles ou styles d’architecture : le modèle à base des
simples fonctions, le modèle RPC, le modèle SOAP, le modèle Rest puis le modèle GraphQL.
Le modèle Rest, est un style orienté ressources et utilisant principalement le protocole HTTP pour la communication. Avec le modèle Rest les données entre le client et le fournisseur sont transmises en format JSON ou
XML. Ainsi, Rest utilise des notions qu’il faut obligatoirement comprendre : les ressources, les identifiants, les méthodes HTTP, et les formats de données.
\subsection{Les principes de l’architecture REST}
L'architecture Rest est régit par six principes suivants:
\begin{enumerate}
	\item La séparation entre client et serveur\\
les responsabilités du côté serveur et du côté client sont
séparées, si bien que chaque côté peut être implémenté indépendamment de l’autre. Le code
côté serveur (l’API) et celui côté client peuvent chacun être modifiés sans affecter l’autre, tant
que tous deux continuent de communiquer dans le même format. Dans une architecture REST,
différents clients envoient des requêtes sur les mêmes endpoints, effectuent les mêmes actions et
obtiennent les mêmes réponses.
     \item . L’absence d’état de sessions (stateless)\\
la communication entre client et serveur ne conserve pas
l’état des sessions d’une requête à l’autre. Autrement dit, l’état d’une session est inclus dans
chaque requête, ce qui signifie que ni le client ni le serveur n’a besoin de connaître l’état de
l’autre pour communiquer. Chaque requête est complète et se suffit à elle-même : pas besoin de maintenir une connexion continue entre client et serveur, ce qui implique une plus
grande tolérance à l’échec. De plus, cela permet aux APIs REST de répondre aux requêtes
de plusieurs clients différents sans saturer les ports du serveur. L’exception à cette règle est
l’authentification, pour que le client n’ait pas à préciser ses informations d’authentification à
chaque requête.
      \item L’uniformité de l’interface\\
les différentes actions et/ou ressources disponibles avec leurs endpoints et leurs paramètres spécifiques doivent être décidés et respectés religieusement, de façon
uniforme par le client et le serveur. Chaque réponse doit contenir suffisamment d’informations
pour être interprétée sans que le client n’ait besoin d’autres informations au préalable. Les
réponses ne doivent pas être trop longues et doivent contenir, si nécessaire, des liens vers d’autres
endpoints. 
       \item La mise en cache\\
les réponses peuvent être mises en cache pour éviter de surcharger inutilement le serveur. La mise en cache doit être bien gérée : l’API REST doit préciser si telle ou telle
réponse peut être mise en cache et pour combien de temps pour éviter que le client ne reçoive
des informations obsolètes.
        \item L’architecture en couches\\
un client connecté à une API REST ne peut en général pas distinguer
s’il est en communication avec le serveur final ou un serveur intermédiaire. Une architecture
REST permet par exemple de recevoir les requêtes sur un serveur A, de stocker ses données sur
un serveur B et de gérer les authentifications sur un serveur C.
        \item Le code à la demande
Cette contrainte est optionnelle. Elle signifie qu’une API peut retourner
du code exécutable au lieu d’une réponse en \textbf{JSON} ou en XML par exemple. Cela signifie qu’une
API \textbf{RESTful} peut étendre le code du client tout en lui simplifiant la vie en lui fournissant du
code exécutable tel qu’un script \textbf{JavaScript}.

\end{enumerate}

Une API REST ne peut être qualifiée de \textbf{RESTful} si elle ne respecte pas les six contraintes, mais on peut
tout de même la qualifier d’API REST si elle n’enfreint que deux ou trois principes. REST est sans doute le
standard le plus utilisé pour concevoir des architectures d’API, mais il en existe bien d’autres qui pourraient le complémenter, voire un jour le détrôner

\section{L’architecture serverless}
C’est une architecture où l’utilisateur n’a pas à gérer la moindre infrastructure. L’architecture serverless ne signifie pas pour autant qu’il n’y a pas de serveurs : cela veut dire qu’ils sont invisibles pour l’utilisateur et sont gérés par les fournisseurs et non par les consommateurs. Sans trop penser à leur maintenance, les ressources informatiques sont utilisées comme des services.
 
L’architecture serverless fait changer la façon dont on conçoit et maintient les applications. Elle permet aux consommateurs de bâtir des plateformes en utilisant exclusivement des services managés, de se concentrer sur les aspects liés à la logique business, et non sur les contraintes de déploiement ou de scalabilité.
\begin{list}{•}
	\item  Exemples fournisseur des architectures Serverless
	\begin{itemize}
		\item AWS
		\begin{list}{*}
			\item Amazon S3
			\item Amazon Dynamodb
			\item Amazon Lambda
        \end{list}
		\item Google:
		\begin{list}{*}
			\item Google Cloud Functions
		\end{list}
		\item Microsoft
		\begin{list}{*}
			\item Azure Functions
		\end{list}
	\end{itemize}

\end{list}


%----------------------------------------------------------------------------------------

\section{Le protocole OAuth2 et la délégation d’autorisation}

Le Oauth2 est la seconde version du protocole Oauth(Open Autorization). Connu comme protocole de
délégation d’autorisations, le Oauth2 est un protocole qui permet à une application d’obtenir une autorisation d’accès limitée aux ressources disponibles sur un serveur accessible via HTTP. Si ces ressources
n’appartiennent pas à l’application, l’autorisation d’accès lui est déléguée par le détenteur des ressources ;
au cas contraire l’application obtient l’autorisation en s’authentifiant avec ses identifiants. Avec le protocole Oauth2 le propriétaire de ressources ne partage pas ses identifiants avec l’application qui sollicite ses
ressources. Le protocole Oauth2 fournit un modèle dans lequel le détenteur de ressource peut accorder un
accès limité à ses données en émettant simplement un jeton temporaire, qui sera utilisé par cette application
sollicitant ses données pour s’identifier auprès du serveur de ressources. Le protocole Oauth2 met le propriétaire de ressources au cœur du système d’octroi d’autorisation. C’est le propriétaire de ressources qui
fait le lien entre ses comptes sur différentes applications sans que des administrateurs de la sécurité aient
besoin d’intervenir directement sur chaque application.

\subsection{Exemple d’implémentation du protocole Oauth2}

Le protocole Oauth2 est utilisé par plusieurs entreprises comme Facebook, Google et Twitter. L’exemple que nous proposons ici est celui qui permet
d’afficher instantanément les tweets sur Facebook sans avoir besoin de votre mot de passe Facebook

\subsection{Le vocabulaire du protocole Oauth2}
\textbf{Les rôles}: Le protocole Oauth2 identifie quatre rôles: Client, Propriétaire de ressources, Serveur d’autorisation
et Serveur de ressources.
\begin{list}{•}
	\item  Le détenteur des ressources(Resource Owner):\\
	 Le détenteur ou le propriétaire de ressources est
	une entité capable d’accorder l’accès à une ressource protégée. Lorsque le propriétaire de la ressource est une personne, on parle d’ utilisateur final.

	\item Le serveur de ressources (Resource Server):\\
	Le serveur de ressources est le serveur qui héberge
	les ressources protégées ou à accès limité. Un exemple du serveur des ressources peut être Facebook et Google qui hébergent les informations des profils des utilisateurs.
	\item Le client (Client Application): \\
	C’est une application qui sollicite les ressources du serveur de
	ressources, celle-ci peut être une application PHP, une application mobile, une application Javascrip.
	\item Le serveur d’autorisation (Authorization Server):\\
	Le serveur d’autorisation représente le serveur
	qui délivre des jetons au client. Ces jetons seront utilisés lors des requêtes du client vers le serveur de ressources. Ce serveur peut être le même que le serveur de ressources (physiquement et applicativement), et c’est souvent le cas.
\end{list}
\textbf{Les jetons:} Un jeton est une chaîne des caractères, générée par le serveur d’autorisation au client une fois
qu’ une autorisation lui est offerte.
\begin{list}{•}
	\item Le jeton d’accès(Access token):\\
	 est un jeton nécessaire pour accéder aux ressources partagées par
	OAuth2. Il a une durée de vie qui est généralement assez courte.
	\item  
	\item Le jeton de renouvellement (Refresh Token):\\
	 est un jeton que le serveur d’autorisation peut émettre aux clients et il peut être échangé contre un nouveau jeton d’accès, sans répéter le processus d’autorisation. Il n’a pas de temps d’expiration.
\end{list}

\textbf{Les types d’autorisations(Grant types)}

\begin{list}{•}
	\item L’autorisation implicite (Implicit Grant): \\
	Elle doit être utilisée quand l’application se trouve côté
	client(typiquement une application Javascript). Il ne permet pas d’obtenir de token de renouvellement.
	
	\begin{figure}[!th]
            \centering
                \includegraphics[width=0.8\textwidth]{Figures/implicit_grant}
	       \decoRule
		\caption[Implicit grant]{Implicit grant}
	\label{fig:implicit}
	\end{figure}
	
	\item L’autorisation via un code (Authorization Code Grant): \\
	Ce type d’autorisation permet d’obtenir deux jetons : le jeton d’accès et le jeton de renouvellement. Le client dans ce cas de figure interagit
	avec le propriétaire des ressources via un client web généralement un navigateur
	\begin{figure}[!th]
            \centering
                \includegraphics[width=0.8\textwidth]{Figures/code_grant}
	       \decoRule
		\caption[Code grant]{Code grant}
	\label{fig:Code}
	\end{figure}
	
	\item L’autorisation serveur à serveur (Client Credentials Grant):\\
	 Elle doit être utilisée lorsque le client est lui-même le détenteur des données. Il n’y a pas d’autorisation à obtenir de la part de l’utilisateur
	 \begin{figure}[!th]
            \centering
                \includegraphics[width=0.8\textwidth]{Figures/server_server}
	       \decoRule
		\caption[Sever grant]{Server grant}
	\label{fig:Server}
	\end{figure}
	
	 \item L’autorisation via mot de passe (Resource Owner Password Credentials Grant):\\
	 Avec ce type d’autorisation, les identifiants sont envoyés au client et ensuite au serveur d’autorisation. Il est donc
	 impératif qu’il y ait une confiance absolue entre ces 2 entités. Ce type d’autorisation est principalement utilisé lorsque le client a été développé par la même autorité que celle          fournissant le serveur d’autorisation final.
	 \begin{figure}[!th]
            \centering
                \includegraphics[width=0.8\textwidth]{Figures/password_grant}
	       \decoRule
		\caption[Password grant]{Password grant}
	\label{fig:Password}
	\end{figure}
\end{list}


%----------------------------------------------------------------------------------------
\section{Intégration de données}
L’intégration des données est le processus qui consiste à combiner des données provenant de différentes
sources dans une vue unifiée : de l’importation au nettoyage en passant par le mapping et la transformation dans un gisement cible, pour finalement rendre les données plus exploitables et plus utiles pour les
utilisateurs qui les consultent.
\subsection{Les avantages}
\begin{list}{•}
	\item L’intégration des données améliore l’unification des systèmes et la collaboration globale
	\item L’intégration des données fait gagner du temps
	\item L’intégration des données réduit les erreurs (et les besoins de modifications)
\end{list}
\subsection{Opérations ETL et intégration des données}
Les opérations d’extraction, de transformation et de chargement (ETL) forment un processus d’intégration
à part entière dans lequel les données sont extraites du système source et livrées au data warehouse. Il s’agit
d’un processus continu que le data warehousing exécute pour transformer plusieurs sources de données en
informations cohérentes et utiles destinées à la Business Intelligence et aux analyses.
 \begin{figure}[!th]
            \centering
                \includegraphics[width=0.8\textwidth]{Figures/etl}
	       \decoRule
		\caption[ETL]{ETL}
	\label{fig:ETL}
	\end{figure}
\section{Conteneurisation}
Un conteneur est un système qui permet de stocker et d’isoler des objets devant être transportés ou déployés dans un environnement d’exploitation étendu (applications, logiciels, librairie, etc.). Il permet également au code applicatif d’être transporté de l’environnement de développement vers celui de production de manière aisée et sûre.  
La conteneurisation permet:
%La conteneurisation permet 

\begin{list}{•}
	\item De virtualiser, à l’intérieur d’un conteneur les ressources matérielles dont une application a besoin pour être exécutée(mémoire, réseau, processeur, etc.) 
	\item
	\item D'embarquer les composants logiciels nécessaires à l’application (données, fichiers, etc.)  
\end{list}

La conteneurisation est parfois confondue avec la virtualisation. Quand la première nécessite que chaque machine virtuelle possède 
son système d’exploitation, les conteneurs eux se connectent au noyau des machines, kernel, pour en exploiter les ressources. 
Moins lourds, ils sont plus faciles à déplacer et à stocker. 
 \begin{figure}[!th]
            \centering
                \includegraphics[width=0.8\textwidth]{Figures/virtualvscont}
	       \decoRule
		\caption[VM vs Conteneurisation]{VM vs Conteneurisation}
   \label{fig:VM vs Conteneurisation}
\end{figure}
%link : https://codalis.ch/conteneurisation-docker/

%Il s’agit d’un type de virtualisation utilisé au niveau des applications. Le principe repose sur la création de
%plusieurs espaces utilisateurs isolés les uns des autres sur un noyau commun. On utilise alors le terme de
%« conteneur » pour désigner une telle instance. Cette séparation repose sur un concept similaire à celui des
%modules applicatifs cloisonnés, communiquant à l’aide de services et applications web. Les conteneurs,
%bien qu’ indépendants, partagent un noyau commun (donc un ou plusieurs systèmes d’exploitation) et un
%même espace mémoire.
%La conteneurisation permet de packager tous les services, scripts, API, librairies dont une application a
%besoin. L’objectif : en permettre l’exécution sur n’importe quel noyau compatible
\subsection{Avantages}
\begin{list}{•}
	\item Elle évite de se soucier d’interactions ou d’incompatibilités avec les conteneurs déjà présents ou à
	venir sur cette machine.
	\item 
	\item Elle permet de ne pas occuper autant de ressources que réclamerait une machine virtuelle (ou virtual machine, VM), qui emporte son propre système d’exploitation et bloque des ressources à son
	lancement.
\end{list}
\subsection{Exemple des systèmes de conteneurisation}
\begin{list}{•}
	\item Docker
	\item Kubernetes
	\item CoreOs rkt
	\item OpenV
\end{list}
 
% Chaptre 1

\chapter{Outils et technologies utilisés} % Main chapter title

\label{Chaptre3} % For referencing the chapter elsewhere, use \ref{Chapter1} 

Dans cette section je vais présenter les outils et les technologies que j’avais utilisé durant mon stage.

%----------------------------------------------------------------------------------------

% Define some commands to keep the formatting separated from the content 
%\newcommand{\keyword}[1]{\textbf{#1}}
%\newcommand{\tabhead}[1]{\textbf{#1}}
%\newcommand{\code}[1]{\texttt{#1}}
%\newcommand{\file}[1]{\texttt{\bfseries#1}}
%\newcommand{\option}[1]{\texttt{\itshape#1}}
%----------------------------------------------------------------------------------------

\section{Amazon Lambda}
AWS Lambda est un service de calcul Serverless qui vous permet d’exécuter du code sans:
\begin{list}{•}
	\item provisionner ou gérer des serveurs,
	\item créer une logique de dimensionnement de cluster prenant en charge la charge de travail,
	\item  maintenir les intégrations d’événements ou gérer les environnements d’exécution.
\end{list}
Avec Lambda, vous pouvez exécuter du code pour pratiquement n’importe quel type d’application ou service backend , sans aucune tâche administrative. Il suffit de télécharger votre code sous forme de fichier
ZIP ou d’image de conteneur, et Lambda alloue automatiquement et précisément la puissance d’exécution
de calcul et exécute votre code en fonction de la demande ou de l’événement entrant, pour n’importe quelle
échelle de trafic.
Vous pouvez configurer votre code de sorte qu’il se déclenche automatiquement depuis plus de 200 applications SaaS et services AWS, ou l’appeler directement à partir de n’importe quelle application web ou
mobile.
Vous pouvez écrire des fonctions Lambda dans votre langage préféré (Node.js, Python, Go, Java, etc.)
\subsection{Fonctionnement}
Chaque fonction Lambda s'exécute dans son propre conteneur. Lorsqu'une fonction est créée, Lambda l'empaquette dans un nouveau conteneur, puis exécute ce conteneur sur un cluster de machines mutualisées géré par AWS. Avant que les fonctions ne commencent à s'exécuter, le conteneur de chaque fonction se voit allouer la mémoire RAM et la capacité CPU nécessaires. Une fois que les fonctions ont fini de s'exécuter, la RAM allouée au début est multipliée par le temps que la fonction a passé à s'exécuter. Les clients sont ensuite facturés en fonction de la mémoire allouée et de la durée d'exécution de la fonction.
\subsection{Avantages}
\begin{list}{•}
	\item Aucun serveur à gérer:
	AWS Lambda exécute automatiquement votre code, sans que vous ayez à mettre en service ou à gérer des
	serveurs
	\item Dimensionnement continu:
	AWS Lambda dimensionne automatiquement votre application en exécutant le code en réponse à chaque
	déclencheur. Votre code s’exécute en parallèle et traite chaque déclencheur indépendamment. La charge de
	travail est ainsi mise à l’échelle de façon précis
	\item Optimisation des coûts grâce au comptage en millisecondes:
	Avec AWS Lambda, vous ne payez que pour le temps de calcul que vous consommez
	\item Performances constantes à n’importe quelle échelle:
	Avec AWS Lambda, vous pouvez optimiser le temps d’exécution de votre code en choisissant la bonne taille de mémoire pour
	votre fonction
\end{list}

%----------------------------------------------------------------------------------------

\section{Docker}
Docker est un système de containérisation le plus utilisé; qui vous permet de créer, déployer et lancer vos applications en utilisant des conteneurs.
Pour mettre en place ces conteneurs, on crée des images Docker. L’image Docker permet de configurer tout l’environnement dans lequel le conteneur va s'exécuter. 
Pour créer ces images, Docker utilise un fichier spécial appelé Dockerfile, qui grâce à une syntaxe simple et élégante va nous permettre de préparer nos images.
L’image est ensuite construite par le démon Docker via l’utilisation de commandes dans le terminal qui sont regroupées dans ce qu’on appelle un CLI.
Pour gérer l’ensemble des conteneurs d’une application, on utilise Docker Compose.

 \begin{figure}[!th]
            \centering
                \includegraphics[width=0.8\textwidth]{Figures/dockerfileimagecontainer}
	       \decoRule
		\caption[Docker]{Docker}
	\label{fig:docker}
\end{figure}
Un conteneur est une instance d'une image et une image est obtenue on compilant le fichier Dockerfile.
\subsection{Docker Hub}
Docker Hub est un service fourni par Docker pour rechercher et partager des images de conteneurs avec votre équipe. 
Il s'agit du plus grand référentiel au monde d'images de conteneurs.
Docker Hub fournit les fonctionnalités principales suivantes :
\begin{list}{•}
	\item Repositories: permet le push et le pull des images des conteneurs
	\item Teams and  Organizations: permet de gérer l'accès aux référentiels privés d'images de conteneurs
	\item Docker Official Images: permet de récupéré  et d'utiliser des images de conteneurs de haute qualité fournies par Docker
	\item Builds: permet de créer automatiquement des images de conteneur à partir de GitHub et Bitbucket et de les transférer vers Docker Hub.
\end{list}
Docker fournit un outil  Docker Hub CLI  et une API qui vous permet d'interagir avec Docker Hub.


\section{Amplify}

AWS Amplify est un ensemble d’outils et de services qui peuvent être utilisés ensemble ou un par un, pour:
\begin{list}{•}
\item Authentification: accéder à des workflows prêts à l'emploi pour MFA, authentification unique, mot de passe oublié, etc.
\item Hébergement: déployer des applications Web statiques en quelques clics et facilement gérer le contenu(JavaScript, React, Angular, Flutter,...)
\item Notifications push: gérer facilement les campagnes et envoyer des messages aux utilisateurs via plusieurs canaux, notamment par SMS, e-mail et push.
\item Analytique: suivre les sessions des utilisateurs et créer des rapports sur leur comportement. Configurer des attributs personnalisés et analyser les entonnoirs de conversion.
\end{list}

%link : https://www.bluematador.com/blog/what-is-aws-amplify
%aider les développeurs web mobile et frontal à créer des applications évolutives et intégrales à technologie
%AWS. Avec Amplify, vous pouvez configurer les backends d’application et connecter votre application en
%quelques minutes, déployer des applications Web statiques en quelques clics et facilement gérer le contenu
%des applications en dehors de la console AWS.
%Amplify prend en charge les frameworks Web populaires, tels que JavaScript, React, Angular, Vue, Next.js,
%et les plateformes mobiles, telles qu’Android, iOS, React Native, Ionic, Flutter.
%\subsubsection{Developpement}
 \begin{figure}[!th]
            \centering
                \includegraphics[width=0.8\textwidth]{Figures/amplify_dif}
	       \decoRule
		\caption[Exemple d'hébergement d'un site static]{Exemple d'hébergement d'un static}
	\label{fig:amplify}
	\end{figure}
%\subsubsection{}
% \begin{figure}[!th]
%            \centering
%              \includegraphics[width=0.8\textwidth]{Figures/amplify-2}
%	       \decoRule
%		\caption[Amplify]{Amplify}
%	\label{fig:amplify}
%	\end{figure}

\section{Amazon Simple Storage Service (Amazon S3)}
Amazon S3 est un stockage d'objets conçu pour stocker et récupérer n'importe quelle quantité de données, n'importe où. Il s'agit d'un service de stockage simple qui offre une durabilité, une disponibilité, des performances, une sécurité, et une scalabilité de pointe pratiquement illimitée à un tarif très bas.
S3 supporte tout type de fichier et peut être utilisé comme repos d'hébergement des sites web statics:
\subsection{Avantages de l'utilisation d'Amazon S3}
Amazon S3 est intentionnellement conçu avec un ensemble de fonctionnalités minimal qui met l'accent sur la simplicité et la robustesse. Voici quelques-uns des avantages de l'utilisation d'Amazon S3 :
\begin{list}{•}
\item \textbf{Création de compartiments:}Créez et nommez un compartiment qui stocke les données. Les compartiments sont les conteneurs fondamentaux d'Amazon S3 pour le stockage de données.
\item \textbf{Stockage de données:}Stockez une quantité infinie de données dans un bucket. Chargez autant d'objets que vous le souhaitez dans un compartiment Amazon S3. Chaque objet peut contenir jusqu'à 5 To de données. Chaque objet est stocké et récupéré à l'aide d'une clé unique attribuée par le développeur.
\item \textbf{Téléchargement de données:}Téléchargez vos données ou permettez à d'autres de le faire. Téléchargez vos données à tout moment ou permettez à d'autres de faire de même.
\item \textbf{Autorisations:}Accordez ou refusez l'accès à d'autres personnes qui souhaitent charger ou télécharger des données dans votre compartiment Amazon S3. Accordez des autorisations de chargement et de téléchargement à trois types d'utilisateurs. Les mécanismes d'authentification peuvent aider à protéger les données contre les accès non autorisés.
\item \textbf{Interfaces standard:}Utilisez des interfaces REST et SOAP basées sur des normes conçues pour fonctionner avec n'importe quelle boîte à outils de développement Internet.
\end{list}


%S3 est un service de stockage d’objet offrant une évolutivité, une disponibilité des données, une sécurité et des performances de pointe. Les clients de toutes tailles et de tous secteurs peuvent ainsi utiliser
%ce service afin de stocker et protéger n’importe quelle quantité de données pour un large éventail de cas
%d’utilisation comme des lacs de données, des sites web, des applications mobiles, la sauvegarde et la restauration, l’archivage, des applications d’entreprise, des appareils IoT et des analyses du Big Data. Amazon
%S3 fournit des fonctions de gestion faciles à utiliser pour vous permettre d’organiser vos données et de
%configurer des contrôles d’accès affinés pour vos exigences métier, d’organisation et de conformité spécifiques. Amazon S3 est conçu pour offrir 99,999999999 \% de durabilité et stocker les données de millions
%d’applications pour des entreprises du monde entier

\section{Aws Aurora}

Amazon Aurora est un moteur de base de données relationnelle qui associe la vitesse et la fiabilité des bases
de données commerciales haut de gamme à la simplicité et la rentabilité des bases de données open source.
Amazon Aurora MySQL offre des performances jusqu’à cinq fois supérieures à celles de MySQL sans
nécessiter de modifications de la plupart des applications MySQL. De la même manière, Amazon Aurora
PostgreSQL offre des performances jusqu’à trois fois supérieures à celles de PostgreSQL. Amazon RDS
gère vos bases de données Amazon Aurora en prenant en charge les tâches chronophages telles que la mise en service, l’application des correctifs, la sauvegarde, la récupération, la détection des pannes, ainsi que
les réparations. Vous payez un forfait mensuel pour chaque instance de base de données Amazon Aurora
utilisée. Aucun coût initial ou engagement à long terme n’est requis.

\subsection{Que signifie « des performances trois fois supérieures à celles de PostgreSQL  et MySQL» }
Amazon Aurora fournit des augmentations significatives des performances de PostgreSQL et  MySQL en intégrant étroitement au moteur de base de données une couche de stockage virtualisée basée sur SSD, conçue principalement pour les charges de travail des bases de données, ce qui permet de réduire les opérations d'écritures dans le système de stockage, minimiser la contention de verrouillage et éliminer les retards créés par les threads de processus de la base de données

\section{Amazon Cognito}
Amazon Cognito permet d’ajouter facilement et rapidement l’inscription et la connexion des utilisateurs
ainsi que le contrôle d’accès aux applications Web et mobiles. Amazon Cognito s’adapte à des millions
d’utilisateurs et prend en charge la connexion avec les fournisseurs d’identité sociale tels qu’Apple, Facebook, Google et Amazon, et les fournisseurs d’identité d’entreprise via SAML 2.0 et OpenID Connect

\subsection{Fonctionnalités}
\subsubsection{Répertoire d’utilisateurs sécuritaire et se mettant à l’échelle}
Les groupes d’utilisateurs d’Amazon Cognito fournissent un répertoire d’utilisateurs sécurisé qui s’étend à
des centaines de millions d’utilisateurs. En tant que service entièrement géré, les groupes d’utilisateurs sont
faciles à configurer sans avoir à s’inquiéter de la mise en place d’une infrastructure serveur.
\subsubsection{Fédération d’identité sociale et d’entreprise}
Avec Amazon Cognito, vos utilisateurs peuvent se connecter via des fournisseurs d’identité sociale tels
que Apple, Google, Facebook et Amazon.

\subsubsection{Authentification basée sur les standards}
Amazon Cognito User Pools est un fournisseur d’identités normalisé et prend en charge les normes de
gestion des identités et des accès, telle que Oauth 2.0.
\subsubsection{Sécurité pour vos applications et vos utilisateurs}
Amazon Cognito prend en charge l’authentification multi-facteurs et le chiffrement des données au repos et
en transit. Amazon Cognito est éligible HIPAA et conforme aux normes PCI DSS, SOC, ISO/IEC 27001,
ISO/IEC 27017, ISO/IEC 27018, et ISO 9001.
\subsubsection{Contrôle d’accès pour les ressources AWS}
Amazon Cognito fournit des solutions pour contrôler l’accès aux ressources AWS à partir de votre application. Vous pouvez définir des rôles et associer des utilisateurs à des rôles différents afin que votre
application puisse accéder uniquement aux ressources autorisées pour chaque utilisateur. Autre possibilité
: vous pouvez également utiliser les attributs des fournisseurs d’identité dans les stratégies d’autorisation
AWS Identity and Access Management. Cela vous permettra de contrôler l’accès à des ressources pour les
utilisateurs qui remplissent des conditions d’attributs spécifiques
\subsubsection{Intégration facile avec votre application}
Amazon Cognito fournit des solutions pour contrôler l’accès aux ressources AWS à partir de votre application. Vous pouvez définir des rôles et associer des utilisateurs à des rôles différents afin que votre
application puisse accéder uniquement aux ressources autorisées pour chaque utilisateur. Autre possibilité
: vous pouvez également utiliser les attributs des fournisseurs d’identité dans les stratégies d’autorisation
AWS Identity and Access Management. Cela vous permettra de contrôler l’accès à des ressources pour les
utilisateurs qui remplissent des conditions d’attributs spécifiques.


\section{API Gateway}

Amazon API Gateway est un service entièrement opéré, qui permet aux développeurs de créer, publier,
gérer, surveiller et sécuriser facilement des API à n’importe quelle échelle. Les API servent de « porte
d’entrée » pour que les applications puissent accéder aux données, à la logique métier ou aux fonctionnalités de vos services backend. À l’aide d’API Gateway, vous pouvez créer des API RESTful et des API
WebSocket qui permettent de concevoir des applications de communication bidirectionnelle en temps réel.
API Gateway prend en charge les charges de travail conteneurisées et sans serveur, ainsi que les applications
web.\\
API Gateway gère toutes les tâches liées à l’acceptation et au traitement de plusieurs centaines de milliers
d’appels d’API simultanés, notamment la gestion du trafic, la prise en charge de CORS, le contrôle des
autorisations et des accès, la limitation, la surveillance et la gestion de la version de l’API. Aucuns frais
minimum ou coûts initiaux ne s’appliquent à API Gateway. Vous payez pour les appels d’API que vous
recevez et la quantité de données transférées et, avec le modèle de tarification par paliers de l’API Gateway,
vous pouvez réduire vos coûts en fonction de l’utilisation de votre API

 \begin{figure}[!th]
            \centering
                \includegraphics[width=0.8\textwidth]{Figures/apigateway}
	       \decoRule
		\caption[API Gateway]{API Gateway}
	\label{fig:apigateway}
	\end{figure}

%----------------------------------------------------------------------------------------

\section{Amazon Dynamodb}
Amazon DynamoDB est une base de données NoSql de type clé-valeur et de documents, offrant des performances de latence de l’ordre de quelques millisecondes, quelle que soit l’échelle. Il s’agit d’une base
de données multi-région, multi-active et durable entièrement gérée, avec des systèmes intégrés de sécurité,
de sauvegarde, de restauration et de mise en cache en mémoire pour les applications à l’échelle d’Internet.
DynamoDB peut traiter plus de 10 mille milliards de demandes par jour et supporte des pics de 20 millions
de demandes par seconde.\\

La plupart des entreprises du monde qui connaissent la croissance la plus rapide, comme Lyft, Airbnb et
Redfin, ainsi que Samsung, Toyota et Capital One s’appuient sur la mise à l’échelle et les performances de
DynamoDB pour prendre en charge leurs charges de travail stratégiques.
Des centaines de milliers de clients AWS ont choisi DynamoDB comme base de données de clés-valeurs et
de documents pour leurs applications mobiles, Web, de jeux, de technologie publicitaire, IoT, etc. nécessitant un accès à faible latence aux données, quelle que soit l’échelle

\section{CloudWatch}
Amazon CloudWatch est un service de surveillance et d'observabilité conçu pour les ingénieurs DevOps, les développeurs, les ingénieurs en fiabilité de sites (SRE) et les responsables informatiques. \\

CloudWatch collecte des données de surveillance et opérationnelles sous forme de journaux, de métriques et d'événements. Ensuite, il les visualise à l'aide de tableaux de bord automatisés pour vous permettre d’avoir une appréciation unifiée de vos ressources, applications et services AWS opérationnels sur AWS et sur site.
 \begin{figure}[!th]
            \centering
                \includegraphics[width=0.8\textwidth]{Figures/cloudwatch}
	       \decoRule
		\caption[CloudWatch]{CloudWatch}
	\label{fig:CloudWatch}
	\end{figure}
\textbf{A compléter}
\section{Reflex}
ReFlex est un système de souscription automatisé modulaire. Il ne s'agit pas d'une application(service web) autonome, elle doit  être intégrée au paysage applicatif du client. Les composants ReFlex requis sont hébergés dans l'environnement du client. Chaque instance de Reflex est fournie avec une base de connaissance paramètrée en fonction des produits d'assurance du client(entreprise d'assurance).
Les principaux modules de Reflex sont:
\begin{list}{•}
\item \textbf{CEP:} Customer Experience Platform
\item
\item \textbf{RAS:} Risk Assessment Service
\item \textbf{DCS:} Document Creation Service 
\end{list}
\subsection{Scénario d'intégration CEP}
En incluant CEP dans le système, toutes les demandes adressées au composant RAS sont filtrées par le backend CEP. Et avant que l'évaluation des risques puisse être lancée, il doit y avoir une toute première étape d'initialisation pour initialiser l'utilisation du CEP. Cette étape comprend l'appel au service d'intégration, qui fait partie du backend CEP, pour créer un jeton Web JSON (JWT) qui sera transmis entre l'interface utilisateur et le backend pour identifier et autoriser l'utilisateur actuel.
 \begin{figure}[!th]
            \centering
                \includegraphics[width=0.8\textwidth]{Figures/cepreflex}
	       \decoRule
		\caption[Scénario d'intégration CEP]{Scénario d'intégration CEP}
\label{fig:Cep}
\end{figure}

% Chaptre 1

\chapter{Approche suivie et solution proposée} % Main chapter title

\label{Chaptre4} % For referencing the chapter elsewhere, use \ref{Chapter1} 

Dans cette section nous ferons une présentation détaillée des solutions que nous avions proposées par rapport à nos missions.

%----------------------------------------------------------------------------------------

% Define some commands to keep the formatting separated from the content 
%\newcommand{\keyword}[1]{\textbf{#1}}
%\newcommand{\tabhead}[1]{\textbf{#1}}
%\newcommand{\code}[1]{\texttt{#1}}
%\newcommand{\file}[1]{\texttt{\bfseries#1}}
%\newcommand{\option}[1]{\texttt{\itshape#1}}
%----------------------------------------------------------------------------------------

\section{Méthodologie de travail: Scrum avec Agile}
\textbf{A compléter}
\section{Les différentes missions et solutions}
\subsection{Mission 1:Etude de faisabilité sur le déploiement d’une application Front-End dans un environnement AWS}
L’application frontend(web app flutter) de Gighamesh est hébergée sur Firebase, tandis que l’essentiel de
ses ressources se trouve dans le cloud d’Amazon. C’est dans le but d’unifier nos environnements cloud que
cette mission m’avait été confiée.
\subsubsection{Observations}
Les applications web Flutter sont  développées avec le langage de programmation Dart. Avant le déploiement d'une application Flutter un build est nécessaire, 
et le résultat de celui-ci est un ensembles des fichiers statiques( html, css, js, ...). Ce sont ces fichiers statiques qui sont déployés.
Pour résoudre  j’avais exploré deux possibilités:
\subsubsection{Solutions}
\subparagraph{Solution1: L’utilisation de S3 comme repos de d'hébergement}
La solution ici consiste à créer un repos S3, de le rendre public, afin de pouvoir accepter tous les trafics.
De configurer Code Pipeline pour les besoins de CI/CD avec Github, De configurer Cloudfront pour la
gestion du trafic et Route 54 pour le routage.
 \begin{figure}[!th]
            \centering
                \includegraphics[width=0.8\textwidth]{Figures/S3}
	       \decoRule
		\caption[Solution à base de S3]{Solution à base de S3}
	\label{fig:S3}
	\end{figure}
%\newpage
\subparagraph{Solution2: L’utilisation d’Amplify comme comme repos d'hébergement}
Cette solution consiste à utiliser le service d’hébergement qu’offre Amplify et son système de CI/CD en
l’associant au système de versioning: Github. J'avais utilisé CloudFront pour la gestion du trafic et Route 54
pour le routage.
NB: C’est la solution 2 qui avait été retenue. Celle-ci est simple et plus adaptée.
 \begin{figure}[!th]
            \centering
                \includegraphics[width=0.8\textwidth]{Figures/amplify}
	       \decoRule
		\caption[Solution à base d'Amplify]{Solution à base d'Amplify}
	\label{fig:Amplify}
	\end{figure}
	
\subsection{Mission 2:Le déploiement et l’intégration du service web (Reflex) dans le parcours de spuscription}
La souscription au produit d’assurance Assurly est subdivisée en trois parcours: P1, P2, et P3. Un utilisateur ne peut se
retrouver que dans un seul parcours en fonction des données fournies. Si un utilisateur se retrouve dans le
parcours P3 un questionnaire plus complexe est nécessaire, c’est à cet instant qu’intervient Reflex.

\subsubsection{Solutions}
%Reflex est composé de trois modules de base: CEP pour frontend, DOCS pour la gestion des rapports et
%RAS qui constitue le cœur du système Reflex.
%La procédure de déploiement consiste mettre à jour le repos Github qui contient les fichiers nécessaires à la
%construction de notre container, à se connecter à notre EC2, à puller les fichiers du repos Github depuis un
%répertoire de notre EC2 puis à construire et déployer notre container avec la commande docker-compose.
\subsubsection{Pipeline de déploiement}
Quand nous recevons de notre partenaire des mises à jour des modules Reflex , nous les déployons(push) sur un repos Github. Notre repos Github est connecté à notre docker hub; ce qui déclenche automatiquement la construction et le déploiement(push) de l'image de notre conteneur sur docker hub. Pour déployer Reflex sur notre EC2: Il suffit soit de récupérer(pull) le contenu du repos Github puis construire l'image et déployer le conteneur ou de récupérer(pull) l'image de docker hub puis déployer le conteneur. La figure ci-dessous illustre le processus.  
 \begin{figure}[!th]
            \centering
                \includegraphics[width=0.8\textwidth]{Figures/pipeline1}
	       \decoRule
		\caption[Pipeline de déploiement de Reflex]{Pipeline de déploiement de Reflex}
	\label{fig:Pipeline de déploiement de Reflex}
\end{figure}

 \begin{figure}[!th]
            \centering
                \includegraphics[width=0.8\textwidth]{Figures/dockerfile}
	       \decoRule
		\caption[Dockerfile de construction de l'image Reflex]{Dockerfile de construction de l'image Reflex}
	\label{fig:Dockerfile de construction de l'image Reflex}
\end{figure}

\subsubsection{Diagramme de séquence intégration}
L'intégration de Reflex sur l'application web est différente de celle sur l'application mobile.

\textbf{Cas de l'application web}
Notre application web est embarquée dans une Iframe, et l'intégration de Reflex passe par une initialisation des cookies. Il nous était impossible d'initialiser les cookies d'une autre domaine depuis l'Iframe de notre application web. Pour palier ce problème nous avons développé une application intermédiaire hébergée sur le même server web que le moteur Reflex. Les utilisateurs en P3 qui passe par notre application web sont redirigés  sur  l'application web intermédiaire, pour soumettre leur questionnaire Reflex et sont redirigés sur l'application web depart à la fin du processus. Le BACKEND sur le diagramme de séquences ci-dessous représente notre infrastructure dans l'environnement AWS.
 \begin{figure}[!th]
            \centering
                \includegraphics[width=0.8\textwidth]{Figures/reflexweb}
	       \decoRule
		\caption[Diagramme d'intégration de Reflex sur l'application web]{Diagramme d'intégration de Reflex sur l'application web}
	\label{fig:Diagramme d'intégration de Reflex sur l'application web}
\end{figure}

\subsection{Mission 3: (Sécurisation d’une API Rest développée avec la technologie Serveless d’AWS (Lambada + python))}
Pour sécuriser une API il faut prendre en compte deux aspects:
\begin{enumerate}
\item Les contrôles d'accès 
\item L'authentification et les autorisations
\end{enumerate}
Il n'ya pas de choix à faire entre les contrôles d'accès et le système d'authentification \& d'autorisation, ils sont complémentaires.
\subsubsection{Authentification et autorisations}
\textbf{Observations}
\begin{enumerate}
\item Dans l'environnement AWS c'est le service Cognito qui gère  les authentifications et les autorisations 
\item Cognito à la base ne peut supporter qu'un seul user pool (support de stockage des utilisateurs) sur une API
\item Nos utilisateurs sont stockés sur plusieurs users
\end{enumerate}

C'est Cognito que nous avons utilisé pour gérer les authentifications et les autorisations. Pour supporter plusieurs user pools sur nos API, nous avons développé un système d'autorisation 
personnalisé à base d'une Lambda. Dans notre implémentation nous avons explorer deux flows: l'implicit grant(en utilisant la web UI de Cognito) et password grant.
\textbf{Implicit grant}
 \begin{figure}[!th]
            \centering
                \includegraphics[width=0.8\textwidth]{Figures/securite}
	       \decoRule
		\caption[L'architecture du système de sécurité, Implicit grant]{L'architecture du système de sécurité, Implicit grant}
	\label{fig:L'architecture du système de sécurité, Implicit grant}
	\end{figure}
\textbf{Password grant}
 \begin{figure}[!th]
            \centering
                \includegraphics[width=0.8\textwidth]{Figures/securite}
	       \decoRule
		\caption[L'architecture du système de sécurité, Password grant]{L'architecture du système de sécurité, Password grant}
	\label{fig:L'architecture du système de sécurité, Password grant}
	\end{figure}
\newpage
\subsection{Mission 4(Mise en place d'un ETL d'intégration de données)}
Gigamesh disposait une quantité importantes des données stockées dans des fichiers txt; des données nécessaires pour effectuer des campagnes de marketing ciblées. Ma mission consistait à intégrer ces données dans une base de données sql pour pouvoir y effectuer facilement  des requêtes.

\subsubsection{Observations}
\begin{enumerate}
\item Tous les fichiers avaient la même structure
\item Le caractère | était utilisé comme séparateur des colonnes
\item Les libellés des colonnes étaient de fois des phrases
\item Les données n'étaient pas typées
\end{enumerate}
\subsubsection{Solution}
Pour résoudre le problème j'avais mise en place un \textbf{ETL}, qui récupère les données brutes(fichiers txt) de leur support de stockage, les transforme avec un \textbf{script Python} et les stocke dans base de données Mysql dans l'environnement AWS via le service Aurora. 

 \begin{figure}[!th]
            \centering
                \includegraphics[width=0.8\textwidth]{Figures/etls}
	       \decoRule
		\caption[Etl d'intégration de données]{Etl d'intégration de données}
	\label{fig:etl}
	\end{figure}
La partie de transformation de mon ETL m'avait servi à la création de la table où sont stockées les données et à leur typage. 
 
%\include{Chapters/Chapter5} 
\chapter*{Conclusion}
%\begin{abstract}
Le stage que j'ai passé chez GigaMesh était très bénéfique pour moi. C'était non seulement un moment d'apprentissage mais aussi d'application. 
Grâce à ce stage j'ai pu découvrir le domaine de l'assurance emprunteur, de travailler avec la méthodologie agile et scrum, de renforcer et d'acquérir des nouvelles compétences dans le domaine de Cloud.\\ \\Ce stage était aussi un moment de confronter mes compétences acquises à l'école à la réalité des entreprise.\\ \\J'ai également par ce stage eu la chance de déployer et d'intégrer le service web Reflex, d'implémenter les protocoles de sécurité sur l'API Gateway, de déployer des applications web dans le cloud d'Amazon et d'intégrer des données sur Mysql dans le service Aurora d'Amazon. 

%GigMesh ambitionne de révolutionner le domaine 
%de l'assurance en proposant un produit d'assurance simple accessible, à juste prix et aligné aux technologies de notre époque. \\ \\
%J'ai passé mon stage chez Gigamesh comme développeur Backend.  Mon travail portait principalement sur des questions liées: au déploiement et l'intégration de microservices, à l'intégration des données, à la sécurisation de l'API et au  déploiement des applications web dans l'environnement AWS. \\ \\
 %J'avais ainsi  amélioré et acquis les compétences techniques liées: au  développement et à la sécurisation
%des API dans l'environnement AWS, à intégration de données dans Mysql avec Aurora, à la construction et au déploiement des conteneur Docker
%et aussi à l’utilisation des machines virtuelles dans le Cloud.\\ \\
%Mon stage chez  Gigamesh m'avait également permis de découvrir le domaine d'assurance, d’acquérir des compétences en gestion de projet avec la méthodologie de travail Scrum et dans l’utilisation de certains outils tel que GitHub et Reflex.
%----------------------------------------------------------------------------------------
%	THESIS CONTENT - APPENDICES
%----------------------------------------------------------------------------------------

%\appendix % Cue to tell LaTeX that the following "chapters" are Appendices

% Include the appendices of the thesis as separate files from the Appendices folder
% Uncomment the lines as you write the Appendices

%\include{Appendices/AppendixA}
%\include{Appendices/AppendixB}
%\include{Appendices/AppendixC}

%----------------------------------------------------------------------------------------
%	BIBLIOGRAPHY
%----------------------------------------------------------------------------------------

%\printbibliography
\chapter*{Bibliographie}
\begin{enumerate}
\item  Rapport de stage de GUYOT Antoine
\item  La web doc d'AWS
\end{enumerate}


%----------------------------------------------------------------------------------------

\end{document}  
